\documentclass[a4paper, 10pt]{article}

\usepackage[utf8]{inputenc}  
\usepackage{/home/omrigan/study/olegstyles}

\title{Задача распределенной оптимизации}
\date{\today}
\author{Oleg Vasilev}

\begin{document}
\maketitle

\section{Постановка задачи}
Задан неориентированный граф коммуникации: в нем $n$ агентов. Задана матрица смежности $A$ и диагональная матрица степеней вершин $Deg$. Опеределена матрица Кирхгофа (Laplacian matrix) $L = Deg - A$ - т. е. сумма по каждой строчке и столбцу равна нулю.

Для каждого агента задана функция $f_i(x)$, и все агенты кооперативно действуют, чтобы оптимизировать сумму этих функций.

\section{Сведение}

Пусть каждый агент хранит свою копию переменных $x_i$. Введем функцию, в которой каждый агент считает свою функцию на своих переменных: $f'(x_1 \dots x_n) = \sum_i f_i(x_i)$.
Тогда можно сформулировать исходную задачу в виде задачи с ограничениями:
\begin{gather}
\min f' \\
Lx = 0 \iff \sqrt{L}x = 0
\end{gather}
Т. е. на сумме функций достигается минимум при том, что очередной раунд коммуникации (домножение на $L$) не может ничего изменить.

Теперь полученную задачу можно свести к неограниченной при помощи аугментированного метода Лагранжа

\begin{gather}
    A(x, u) = f'(x_1 \dots x_n) + \frac{\alpha}{2} \sum_{i, j} x_i \cdot x_jL_{ij} + \beta u^T \sqrt{L}x
\end{gather}
Таким образом, переменные $u$ можно рассматривать как двойственные - они показывают насколько то или иное ограничение выполннено.

Теперь можно решить полученную задачу прямо-двойственным алгоритмом, выполняя шаги одновременно по $x$ и $u$. По $x$ это будет шаг градиентного спуска, а по $v$ - подъема в сторону увеличения важности насыщенных ограничений в выражении $A(x, u)$. 

Можно выписать формулы пересчета  переменных для $i$-того агента:

\begin{gather}
    v_i = \sqrt{L} u_i \\
    x_{i} \leftarrow x_i - \eta(\alpha\sum_{j} L_{ij} x_{j} + \beta v_i + \nabla f_i(x_i)) \\
    v_{i} \leftarrow v_i + \eta\beta(\sum_j L_{ij}x_j)
\end{gather}


Удивительным образом выражение для пересчета переменных зависит только от переменных, которые принадлежат "соседним" агентам - то есть тем, для которых в соответствующей строке $L$ стоит не-ноль.

Далее, если специальным образом выбрать $\alpha$ и $\beta$, то можно добиться линейной скорости сходимости (т. е. очередное значение переменной в константу раз ближе к оптимуму). Константа будет зависить от матричных свойств $L$, т. е. чем сложнее добраться от одной части графа до другой, тем хуже будет сходимость.

\section{Альтернативное сведение}

Можно использовать альтернативный способ сведения задачи с ограничениями к задаче без ограничений

\begin{gather}
    \arg \min f(x) + \frac{R_y^2}{\varepsilon} ||Ax||^2_2 \\ 
    \arg \min \sum_i f(x_i) + \frac{R_y^2}{\varepsilon} \sum_i \sum_k \left(\sum_j L_{ij} x_j^{(k)}\right)^2 \\ 
    x_i^{(k)} \leftarrow \nabla^{(k)} f(x_i) + \frac{2R_y^2}{\varepsilon} 
        \left(\sum_j L_{ij} x_j^{(k)} \right)L_{ii} \\
    x_i \leftarrow \nabla f(x_i) + \frac{2R_y^2}{\varepsilon} 
        \left(\sum_j L_{ij} x_j \right)L_{ii}
\end{gather}

\end{document}
